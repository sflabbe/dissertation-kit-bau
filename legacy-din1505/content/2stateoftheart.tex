% Chapter 2: State of the Art

\chapter{State of the Art}
\label{ch:state_of_art}

\section{Introduction}
\label{sec:sota_intro}

This chapter provides an overview of existing research on numerical modeling and simulation approaches. The focus is on analytical and computational methods relevant to the problem at hand.

\section{Analytical Approaches}
\label{sec:analytical}

Early work was based on analytical solutions. Classical approaches treat the system using simplified assumptions \cite{Smith2010, Johnson2015}.

The basic equation of motion can be written as:
\begin{equation}
	m\ddot{x} + c\dot{x} + kx = F(t)
	\label{eq:eom_basic}
\end{equation}

where $m$ is mass, $c$ is damping, $k$ is stiffness, and $F(t)$ is the external force.

\section{Numerical Methods}
\label{sec:numerical}

\subsection{Finite Element Methods}
\label{subsec:fem}

Finite element methods allow detailed representation of complex geometries \cite{Wilson2012}. These methods discretize the domain into elements and solve the weak form of the governing equations.

\subsection{Time Integration Schemes}
\label{subsec:integration}

Common time integration methods include:
\begin{itemize}
	\item Explicit methods (e.g., central difference)
	\item Implicit methods (e.g., Newmark-$\beta$)
	\item Adaptive time stepping approaches
\end{itemize}

\section{Parameter Sensitivity}
\label{sec:sensitivity}

Previous studies have investigated the influence of various parameters on system response \cite{Brown2013, Davis2016}. Key parameters include:
\begin{itemize}
	\item Material properties
	\item Boundary conditions
	\item Loading characteristics
	\item Geometric parameters
\end{itemize}

\section{Summary and Research Gaps}
\label{sec:sota_summary}

While significant progress has been made, several challenges remain:
\begin{enumerate}
	\item Need for computationally efficient models
	\item Better validation across different conditions
	\item Development of practical design tools
\end{enumerate}

This work addresses these gaps by developing an efficient model combining accuracy with computational performance.

% Chapter 4: Modeling Approach

\chapter{Modeling Approach}
\label{ch:modeling}

\section{Introduction}
\label{sec:modeling_intro}

This chapter presents the development of the discrete model for predicting impact loads from railway traffic. The model combines multi-body dynamics for the vehicle with a discrete track representation, connected through nonlinear contact mechanics.

\section{Model Architecture}
\label{sec:architecture}

The overall model consists of three main components:

\begin{enumerate}
	\item \textbf{Vehicle model:} Multi-body representation with lumped masses and springs
	\item \textbf{Track model:} Discrete support model with periodic structure
	\item \textbf{Contact model:} Nonlinear spring representing wheel-rail interaction
\end{enumerate}

\section{Vehicle Model}
\label{sec:vehicle_model}

\subsection{Equations of Motion}
\label{subsec:vehicle_eom}

The vehicle is modeled as a multi-body system with the following degrees of freedom:

\begin{itemize}
	\item Car body vertical displacement: $z_c$
	\item Bogie vertical displacements: $z_{b1}$, $z_{b2}$
	\item Wheelset vertical displacements: $z_{w1}$, $z_{w2}$, $z_{w3}$, $z_{w4}$
\end{itemize}

The equations of motion for the car body are:

\begin{equation}
	m_c \ddot{z}_c + c_s (\dot{z}_c - \dot{z}_{b1}) + k_s (z_c - z_{b1}) + c_s (\dot{z}_c - \dot{z}_{b2}) + k_s (z_c - z_{b2}) = 0
	\label{eq:car_body}
\end{equation}

For each bogie:

\begin{equation}
	\begin{split}
		m_b \ddot{z}_{bi} &+ c_s (\dot{z}_{bi} - \dot{z}_c) + k_s (z_{bi} - z_c) \\
		&+ c_p (\dot{z}_{bi} - \dot{z}_{w,2i-1}) + k_p (z_{bi} - z_{w,2i-1}) \\
		&+ c_p (\dot{z}_{bi} - \dot{z}_{w,2i}) + k_p (z_{bi} - z_{w,2i}) = 0
	\end{split}
	\label{eq:bogie}
\end{equation}

For each wheelset:

\begin{equation}
	m_w \ddot{z}_{wi} + c_p (\dot{z}_{wi} - \dot{z}_{bj}) + k_p (z_{wi} - z_{bj}) + F_{ci} = 0
	\label{eq:wheelset}
\end{equation}

where $F_{ci}$ is the contact force between wheel $i$ and the rail.

\subsection{State-Space Formulation}
\label{subsec:state_space}

The system can be written in state-space form as:

\begin{equation}
	\mathbf{M} \ddot{\mathbf{z}} + \mathbf{C} \dot{\mathbf{z}} + \mathbf{K} \mathbf{z} = \mathbf{F}_c
	\label{eq:state_space}
\end{equation}

where $\mathbf{M}$, $\mathbf{C}$, and $\mathbf{K}$ are the mass, damping, and stiffness matrices, and $\mathbf{F}_c$ is the vector of contact forces.

\section{Track Model}
\label{sec:track_model}

\subsection{Discrete Support Model}
\label{subsec:discrete_track}

The track is modeled as a beam on discrete supports representing the sleepers. The equation of motion for the rail is:

\begin{equation}
	EI \frac{\partial^4 w}{\partial x^4} + \mu \frac{\partial^2 w}{\partial t^2} = -\sum_{i=1}^{N} F_i \delta(x - x_i) + F_c(x,t)
	\label{eq:rail}
\end{equation}

where:
\begin{itemize}
	\item $EI$ is the bending stiffness of the rail
	\item $\mu$ is the mass per unit length
	\item $F_i$ are the support forces at sleeper positions $x_i$
	\item $F_c(x,t)$ is the wheel-rail contact force
\end{itemize}

\subsection{Support Characteristics}
\label{subsec:support}

Each support (sleeper) is modeled as a spring-damper-mass system:

\begin{equation}
	m_s \ddot{z}_s + c_s \dot{z}_s + k_s z_s = F_{\text{rail}}
	\label{eq:support}
\end{equation}

The support stiffness includes contributions from:
\begin{itemize}
	\item Rail pad stiffness: $k_{\text{pad}}$
	\item Ballast stiffness: $k_{\text{ballast}}$
	\item Subgrade stiffness: $k_{\text{sub}}$
\end{itemize}

Combined in series:
\begin{equation}
	\frac{1}{k_s} = \frac{1}{k_{\text{pad}}} + \frac{1}{k_{\text{ballast}}} + \frac{1}{k_{\text{sub}}}
	\label{eq:series_stiffness}
\end{equation}

\section{Wheel-Rail Contact Model}
\label{sec:contact_model}

\subsection{Hertzian Contact}
\label{subsec:hertz_contact}

The wheel-rail contact is modeled using Hertzian contact theory. The contact force is:

\begin{equation}
	F_c = \begin{cases}
		C_H \delta^{3/2} & \text{if } \delta > 0 \\
		0 & \text{if } \delta \leq 0
	\end{cases}
	\label{eq:contact_force}
\end{equation}

where $\delta$ is the contact deformation and $C_H$ is the Hertzian contact coefficient:

\begin{equation}
	C_H = \frac{4}{3} E^* \sqrt{R}
	\label{eq:hertz_coeff}
\end{equation}

with equivalent elastic modulus:
\begin{equation}
	\frac{1}{E^*} = \frac{1-\nu_1^2}{E_1} + \frac{1-\nu_2^2}{E_2}
	\label{eq:equiv_modulus}
\end{equation}

\subsection{Contact Deformation}
\label{subsec:contact_def}

The contact deformation is calculated as:

\begin{equation}
	\delta = z_w - w_{\text{rail}}(x_w, t) - r(x_w)
	\label{eq:contact_deformation}
\end{equation}

where:
\begin{itemize}
	\item $z_w$ is the wheelset vertical position
	\item $w_{\text{rail}}$ is the rail deflection at the wheel position
	\item $r(x_w)$ is the combined wheel and rail irregularity profile
\end{itemize}

\section{Track Irregularities}
\label{sec:irregularities}

\subsection{Spectral Representation}
\label{subsec:spectral}

Track irregularities are generated using power spectral density (PSD) functions. The one-sided PSD is given by:

\begin{equation}
	S(\Omega) = \frac{A \Omega_c^2}{(\Omega^2 + \Omega_r^2)(\Omega^2 + \Omega_c^2)}
	\label{eq:psd}
\end{equation}

where:
\begin{itemize}
	\item $A$ is the roughness coefficient
	\item $\Omega_c$ is the cut-off angular spatial frequency
	\item $\Omega_r$ is the reference angular spatial frequency
\end{itemize}

\subsection{Time-Domain Generation}
\label{subsec:time_domain}

Irregularity profiles in the time domain are generated using inverse Fourier transform with random phase angles:

\begin{equation}
	r(x) = \sum_{i=1}^{N} \sqrt{2 S(\Omega_i) \Delta\Omega} \cos(\Omega_i x + \phi_i)
	\label{eq:irregularity_profile}
\end{equation}

where $\phi_i$ are uniformly distributed random phase angles.

\section{Numerical Implementation}
\label{sec:implementation}

\subsection{Time Integration}
\label{subsec:integration}

The equations of motion are integrated using the Newmark-$\beta$ method with parameters $\beta = 0.25$ and $\gamma = 0.5$ (constant acceleration method):

\begin{align}
	\mathbf{z}_{n+1} &= \mathbf{z}_n + \Delta t \dot{\mathbf{z}}_n + \frac{\Delta t^2}{2}[(1-2\beta)\ddot{\mathbf{z}}_n + 2\beta\ddot{\mathbf{z}}_{n+1}] \\
	\dot{\mathbf{z}}_{n+1} &= \dot{\mathbf{z}}_n + \Delta t[(1-\gamma)\ddot{\mathbf{z}}_n + \gamma\ddot{\mathbf{z}}_{n+1}]
	\label{eq:newmark}
\end{align}

\subsection{Contact Detection}
\label{subsec:contact_detection}

At each time step, the contact deformation is evaluated. If $\delta > 0$, the wheel is in contact and the force is calculated using Equation~\ref{eq:contact_force}. If $\delta \leq 0$, separation occurs and $F_c = 0$.

\subsection{Computational Efficiency}
\label{subsec:efficiency}

The model achieves computational efficiency through:
\begin{itemize}
	\item Discrete representation avoiding continuous FE discretization
	\item Efficient handling of periodic track structure
	\item Adaptive time stepping based on contact state
	\item Optimized linear algebra operations
\end{itemize}

Typical computation times are less than 1 minute for 100~m of track on a standard desktop computer.

\section{Model Validation Strategy}
\label{sec:validation_strategy}

The model is validated through:
\begin{enumerate}
	\item Comparison with analytical solutions for simple cases
	\item Verification against published simulation results
	\item Validation with experimental field measurements
	\item Sensitivity analysis and parameter studies
\end{enumerate}

Details of the validation are presented in Chapter~\ref{ch:examples}.

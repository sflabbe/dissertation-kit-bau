% German Abstract (Kurzfassung)

\chapter*{Kurzfassung}

Diese Arbeit befasst sich mit der Entwicklung eines diskreten Modells zur Vorhersage von Sto\ss{}lasten aus dem Schienenverkehr. Die durch den Zugverkehr verursachten dynamischen Lasten stellen eine erhebliche Herausforderung f\"ur die Infrastruktur dar, insbesondere bei h\"oheren Geschwindigkeiten und erh\"ohten Achslasten moderner Z\"uge.

Das entwickelte Modell basiert auf einem diskreten Ansatz, der die Wechselwirkung zwischen Fahrzeug und Gleis ber\"ucksichtigt. Dabei werden folgende Aspekte untersucht:

\begin{itemize}
	\item Diskretisierung des Fahrzeug-Gleis-Systems
	\item Modellierung der Radunrundheiten und Gleisunregelm\"a\ss{}igkeiten
	\item Ber\"ucksichtigung nichtlinearer Kontaktbedingungen
	\item Einfluss verschiedener Fahrgeschwindigkeiten
	\item Auswirkungen unterschiedlicher Fahrzeugtypen
\end{itemize}

Die Validierung des Modells erfolgt durch Vergleich mit experimentellen Daten aus Feldmessungen. Die Ergebnisse zeigen eine gute \"Ubereinstimmung zwischen Simulation und Messung, insbesondere im Frequenzbereich der relevanten Sto\ss{}lasten.

Das entwickelte Modell erm\"oglicht eine effiziente Vorhersage der zu erwartenden Belastungen und kann als Grundlage f\"ur die Dimensionierung von Gleisstrukturen sowie f\"ur Wartungsstrategien dienen. Die Berechnungseffizienz erlaubt die Durchf\"uhrung umfangreicher Parameterstudien zur Optimierung der Gleisinfrastruktur.

\textbf{Schl\"usselw\"orter:} Schienenverkehr, Sto\ss{}lasten, Diskrete Modellierung, Fahrzeug-Gleis-Interaktion, Dynamische Lasten

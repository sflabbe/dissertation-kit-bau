% English Abstract

\chapter*{Abstract}

This work addresses the development of a discrete model for predicting impact loads from railway traffic. The dynamic loads caused by train traffic pose a significant challenge for infrastructure, particularly at higher speeds and increased axle loads of modern trains.

The developed model is based on a discrete approach that considers the interaction between vehicle and track. The following aspects are investigated:

\begin{itemize}
	\item Discretization of the vehicle-track system
	\item Modeling of wheel irregularities and track irregularities
	\item Consideration of nonlinear contact conditions
	\item Influence of different train speeds
	\item Effects of different vehicle types
\end{itemize}

The validation of the model is performed by comparison with experimental data from field measurements. The results show good agreement between simulation and measurement, particularly in the frequency range of relevant impact loads.

The developed model enables efficient prediction of expected loads and can serve as a basis for the dimensioning of track structures as well as for maintenance strategies. The computational efficiency allows the execution of extensive parameter studies for the optimization of track infrastructure.

Furthermore, the model provides insights into the mechanisms of load generation and transmission in the vehicle-track system. This understanding is essential for developing improved design guidelines and for assessing the long-term performance of railway infrastructure under realistic operating conditions.

\textbf{Keywords:} Railway traffic, Impact loads, Discrete modeling, Vehicle-track interaction, Dynamic loads

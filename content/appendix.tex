% Appendix

\chapter{Appendix}
\label{ch:appendix}

\section{Mathematical Derivations}
\label{sec:math_derivations}

\subsection{Derivation of Beam on Elastic Foundation Solution}
\label{subsec:beam_foundation_derivation}

Starting from the Euler-Bernoulli beam equation with elastic foundation:

\begin{equation}
	EI \frac{\partial^4 w}{\partial x^4} + k_f w = F(x,t)
	\label{eq:beam_eq_app}
\end{equation}

For a moving harmonic load $F(x,t) = F_0 \delta(x - vt)$, using the transformation $\xi = x - vt$:

\begin{equation}
	EI \frac{d^4 w}{d\xi^4} + k_f w = F_0 \delta(\xi)
	\label{eq:transformed}
\end{equation}

The characteristic equation is:
\begin{equation}
	EI\lambda^4 + k_f = 0
	\label{eq:characteristic}
\end{equation}

with roots:
\begin{equation}
	\lambda_{1,2,3,4} = \pm\beta(1 \pm i), \quad \beta = \sqrt[4]{\frac{k_f}{4EI}}
	\label{eq:roots}
\end{equation}

The general solution for $\xi > 0$ is:
\begin{equation}
	w(\xi) = e^{-\beta\xi}(C_1\cos\beta\xi + C_2\sin\beta\xi)
	\label{eq:general_solution}
\end{equation}

Applying boundary conditions and continuity requirements yields the deflection and bending moment distributions.

\subsection{Hertzian Contact Stiffness Derivation}
\label{subsec:hertz_derivation}

For two elastic bodies in contact with radii $R_1$ and $R_2$, the equivalent radius is:

\begin{equation}
	\frac{1}{R} = \frac{1}{R_1} + \frac{1}{R_2}
	\label{eq:equiv_radius}
\end{equation}

The Hertzian theory gives the relationship between normal force $F_N$ and approach $\delta$:

\begin{equation}
	\delta = \left(\frac{9F_N^2}{16E^{*2}R}\right)^{1/3}
	\label{eq:hertz_approach}
\end{equation}

Inverting this relationship:

\begin{equation}
	F_N = \frac{4}{3}E^*\sqrt{R}\delta^{3/2} = C_H\delta^{3/2}
	\label{eq:hertz_force_app}
\end{equation}

The linearized contact stiffness at a given load $F_0$ is:

\begin{equation}
	k_c = \frac{dF_N}{d\delta}\bigg|_{\delta_0} = \frac{3}{2}C_H\delta_0^{1/2} = \frac{3}{2}\frac{F_0}{\delta_0}
	\label{eq:linearized_stiffness}
\end{equation}

\section{Model Parameters}
\label{sec:model_parameters}

\subsection{Standard Vehicle Parameters}
\label{subsec:vehicle_parameters}

\begin{table}[htbp]
	\centering
	\caption{Standard vehicle parameters used in simulations}
	\label{tab:vehicle_params_app}
	\begin{tabular}{llcc}
		\toprule
		Parameter & Symbol & Passenger & Freight \\
		\midrule
		Car body mass & $m_c$ [kg] & 30,000 & 40,000 \\
		Bogie mass & $m_b$ [kg] & 3,000 & 4,500 \\
		Wheelset mass & $m_w$ [kg] & 1,500 & 2,200 \\
		Secondary stiffness & $k_s$ [N/m] & $3 \times 10^5$ & $5 \times 10^5$ \\
		Primary stiffness & $k_p$ [N/m] & $1 \times 10^6$ & $8 \times 10^5$ \\
		Secondary damping & $c_s$ [Ns/m] & 30,000 & 40,000 \\
		Primary damping & $c_p$ [Ns/m] & 20,000 & 30,000 \\
		Wheelset radius & $R_w$ [m] & 0.46 & 0.50 \\
		Axle load & $F_0$ [kN] & 100 & 225 \\
		\bottomrule
	\end{tabular}
\end{table}

\subsection{Standard Track Parameters}
\label{subsec:track_parameters}

\begin{table}[htbp]
	\centering
	\caption{Standard track parameters used in simulations}
	\label{tab:track_params_app}
	\begin{tabular}{llc}
		\toprule
		Parameter & Symbol & Value \\
		\midrule
		Rail profile & & UIC 60 \\
		Rail mass per length & $\mu$ [kg/m] & 60 \\
		Rail bending stiffness & $EI$ [MNm$^2$] & 6.45 \\
		Sleeper spacing & $l_s$ [m] & 0.60 \\
		Sleeper mass & $m_s$ [kg] & 280 \\
		Rail pad stiffness & $k_{\text{pad}}$ [kN/mm] & 600 \\
		Rail pad damping & $c_{\text{pad}}$ [kNs/m] & 60 \\
		Ballast stiffness & $k_{\text{ballast}}$ [kN/mm] & 120 \\
		Ballast damping & $c_{\text{ballast}}$ [kNs/m] & 80 \\
		Subgrade stiffness & $k_{\text{sub}}$ [kN/mm] & 200 \\
		Subgrade damping & $c_{\text{sub}}$ [kNs/m] & 40 \\
		\bottomrule
	\end{tabular}
\end{table}

\section{Numerical Methods}
\label{sec:numerical_methods}

\subsection{Newmark Integration Scheme}
\label{subsec:newmark_scheme}

The Newmark-$\beta$ method uses the following update equations:

\paragraph{Displacement predictor:}
\begin{equation}
	\tilde{\mathbf{z}}_{n+1} = \mathbf{z}_n + \Delta t \dot{\mathbf{z}}_n + \frac{\Delta t^2}{2}(1-2\beta)\ddot{\mathbf{z}}_n
	\label{eq:disp_predictor}
\end{equation}

\paragraph{Velocity predictor:}
\begin{equation}
	\tilde{\dot{\mathbf{z}}}_{n+1} = \dot{\mathbf{z}}_n + \Delta t(1-\gamma)\ddot{\mathbf{z}}_n
	\label{eq:vel_predictor}
\end{equation}

\paragraph{Effective stiffness matrix:}
\begin{equation}
	\mathbf{K}_{\text{eff}} = \mathbf{K} + \frac{\gamma}{\beta\Delta t}\mathbf{C} + \frac{1}{\beta\Delta t^2}\mathbf{M}
	\label{eq:eff_stiffness}
\end{equation}

\paragraph{Effective force vector:}
\begin{equation}
	\mathbf{F}_{\text{eff}} = \mathbf{F}_{n+1} + \mathbf{M}\left(\frac{1}{\beta\Delta t^2}\tilde{\mathbf{z}}_{n+1}\right) + \mathbf{C}\left(\frac{\gamma}{\beta\Delta t}\tilde{\mathbf{z}}_{n+1} - \tilde{\dot{\mathbf{z}}}_{n+1}\right)
	\label{eq:eff_force}
\end{equation}

\paragraph{Acceleration corrector:}
\begin{equation}
	\ddot{\mathbf{z}}_{n+1} = \mathbf{K}_{\text{eff}}^{-1}\mathbf{F}_{\text{eff}}
	\label{eq:acc_corrector}
\end{equation}

\paragraph{Final updates:}
\begin{align}
	\mathbf{z}_{n+1} &= \tilde{\mathbf{z}}_{n+1} + \beta\Delta t^2\ddot{\mathbf{z}}_{n+1} \\
	\dot{\mathbf{z}}_{n+1} &= \tilde{\dot{\mathbf{z}}}_{n+1} + \gamma\Delta t\ddot{\mathbf{z}}_{n+1}
	\label{eq:final_updates}
\end{align}

\subsection{Time Step Selection}
\label{subsec:timestep}

The time step is selected based on the stability criterion and accuracy requirements:

\begin{equation}
	\Delta t \leq \frac{T_{\min}}{n_{\text{steps}}}
	\label{eq:timestep_criterion}
\end{equation}

where $T_{\min}$ is the shortest period of interest and $n_{\text{steps}}$ is the number of steps per period (typically 20--50).

For the contact problem, the highest relevant frequency is typically the wheelset bouncing frequency:

\begin{equation}
	f_w = \frac{1}{2\pi}\sqrt{\frac{k_c + k_p}{m_w}}
	\label{eq:wheelset_freq}
\end{equation}

A time step of $\Delta t = 0.0001$~s is typically sufficient for speeds up to 300~km/h.

\section{Track Irregularity Spectra}
\label{sec:irregularity_spectra}

\subsection{Standard PSD Functions}
\label{subsec:psd_functions}

The track irregularity PSD according to ORE/ERRI specification:

\begin{equation}
	S_v(\Omega) = \frac{A_v \Omega_c^2}{(\Omega^2 + \Omega_r^2)(\Omega^2 + \Omega_c^2)}
	\label{eq:ore_psd}
\end{equation}

with parameters:
\begin{itemize}
	\item $\Omega_r = 0.8246$ rad/m (wavelength 7.62~m)
	\item $\Omega_c = 0.0206$ rad/m (wavelength 305~m)
\end{itemize}

Quality levels:

\begin{table}[htbp]
	\centering
	\caption{Track quality levels (ORE specification)}
	\label{tab:quality_levels}
	\begin{tabular}{lcc}
		\toprule
		Quality level & $A_v$ [rad$^2$/m] & Description \\
		\midrule
		Very good & $0.25 \times 10^{-6}$ & New high-speed track \\
		Good & $1.0 \times 10^{-6}$ & Well-maintained conventional track \\
		Average & $4.0 \times 10^{-6}$ & Normal conventional track \\
		Poor & $16.0 \times 10^{-6}$ & Degraded track requiring maintenance \\
		\bottomrule
	\end{tabular}
\end{table}

\section{Measurement Data Processing}
\label{sec:data_processing}

\subsection{Signal Filtering}
\label{subsec:filtering}

Measured force signals are filtered using a Butterworth band-pass filter:

\begin{itemize}
	\item Lower cutoff frequency: 1~Hz (remove quasi-static component)
	\item Upper cutoff frequency: 500~Hz (remove high-frequency noise)
	\item Filter order: 4 (roll-off 80~dB/decade)
\end{itemize}

\subsection{Statistical Metrics}
\label{subsec:statistics}

The following statistical metrics are used for comparison:

\paragraph{Mean value:}
\begin{equation}
	\bar{F} = \frac{1}{N}\sum_{i=1}^{N}F_i
	\label{eq:mean}
\end{equation}

\paragraph{RMS value:}
\begin{equation}
	F_{\text{RMS}} = \sqrt{\frac{1}{N}\sum_{i=1}^{N}(F_i - \bar{F})^2}
	\label{eq:rms}
\end{equation}

\paragraph{Peak factor:}
\begin{equation}
	\text{PF} = \frac{F_{\max}}{F_{\text{RMS}}}
	\label{eq:peak_factor}
\end{equation}

\paragraph{Percentile values:}
$P_{95}$ denotes the value exceeded by only 5\% of the samples.

\section{Software Implementation}
\label{sec:software}

\subsection{Programming Language}
\label{subsec:programming}

The model is implemented in MATLAB/Octave for ease of development and visualization. Critical computational kernels are implemented in C++ for performance.

\subsection{Code Structure}
\label{subsec:code_structure}

Main components:
\begin{itemize}
	\item \texttt{VehicleModel.m}: Vehicle dynamics model
	\item \texttt{TrackModel.m}: Track structure model
	\item \texttt{ContactModel.m}: Wheel-rail contact calculations
	\item \texttt{IrregularityGenerator.m}: Track irregularity generation
	\item \texttt{Integrator.m}: Time integration routines
	\item \texttt{PostProcessor.m}: Results analysis and visualization
\end{itemize}

\subsection{Computational Performance}
\label{subsec:performance}

Typical performance on standard hardware (Intel i7, 16GB RAM):

\begin{table}[htbp]
	\centering
	\caption{Computational performance}
	\label{tab:performance}
	\begin{tabular}{lcc}
		\toprule
		Simulation case & Track length & Computation time \\
		\midrule
		Simple vehicle, smooth track & 100~m & 15~s \\
		Full vehicle, irregular track & 100~m & 45~s \\
		Full vehicle, irregular track & 1000~m & 6~min \\
		Parameter study (100 cases) & 100~m each & 75~min \\
		\bottomrule
	\end{tabular}
\end{table}

\section{Additional Figures}
\label{sec:additional_figures}

This section would contain additional figures and plots that support the main text but are too detailed for inclusion in the main chapters.

% Placeholder figures would go here in a real dissertation

\section{List of Publications}
\label{sec:publications}

Publications resulting from this dissertation work:

\begin{enumerate}
	\item Author Name (Year). "Article Title." \textit{Journal Name}, Vol. X, No. Y, pp. ZZ--ZZ.
	\item Author Name, Co-Author (Year). "Conference Paper Title." \textit{Proceedings of Conference Name}, Location, pp. ZZ--ZZ.
\end{enumerate}
